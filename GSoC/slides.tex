\documentclass{beamer}
\usetheme{AnnArbor}

\begin{document}

\title{Google Summer Of Code}
\author{Saurabh Sood}

\begin{frame}
\titlepage
\end{frame}

\begin{frame}{Table Of Contents}
\tableofcontents
\end{frame}

\section{Who are we?}
\begin{frame}{Know the Devils}

\end{frame}


\section{What is Google Summer Of Code?}
\subsection{Introduction}
\begin{frame}{What is it}
An internship program for students enrolled in an University
\end{frame}

\subsection{Goals}
\begin{frame}{Goals Of the program?}
\pause
\begin{itemize}
\item Highlight Open Source Culture among students \pause
\item Introduce students to Real life Software development \pause
\item Get more open source created \pause
\item Students get paid a handsome stipend of 5000 USD 
\end{itemize}
\end{frame}

\section{History of GSoC}
\begin{frame}{History}\pause
\begin{itemize}
\item Started in 2005 to give back to the open source community \pause
\item openSUSE First Participated in 2006 \pause
\item We did not participate in 2010 \pause
\item Other than that we have participated all the years 
\end{itemize}
\end{frame}

\section{Statistics}
\begin{frame}{openSUSE in numbers}
Success : Failure Ratio
\begin{itemize}
\pause
\item 2009 : 6:9
\item 2011 : 13:14
\item 2012 : 9:12
\item 2013 : 10:12
\item 2014 : 14/14
\end{itemize}
\end{frame}

\section{How does it work?}
\begin{frame}{How does it work?}
\pause
Google chooses mentoring organizations to participate in the program at the start of the year. \pause Every year 160-180 organizations are selected by Google to mentor students
\end{frame}

\subsection{Who is eligible to participate?}
\begin{frame}{Eligibility}
\pause
\begin{itemize}
\item Students over the age of 18 \pause
\item Students enrolled in an University program pursuing their Bachelors, Masters or Doctorates \pause
\item Students who are not residents of countries that are prohibited by US Govt. 
\end{itemize}
\end{frame}


\subsection{Typical Timeline}
\begin{frame}{A guestimate of a timeline}
\begin{itemize}
\pause
\item Starts from March and ends by September \pause
\item In March, Organizations submit their application to Google \pause
\item Idea page is very important in this, the way it is organized \pause
\item Google Reviews organization applicaitions and annouces organizations \pause
\item Students start applying to the organization, get to know the organization and explore projects \pause
\item They talk to their mentors and form their proposals by the end of April \pause
\item Student Selection results are announced by Mid-May \pause
\item Coding period begins in the last week of May \pause
\item Mid Term Evaluations \pause
\item By the first week of August, students put their pencil down and get their code into the repository \pause
\item Mentors attend mentor summit
\end{itemize}
\end{frame}

\section{Being a student}
\begin{frame}{How to apply?}
\begin{itemize}
\item Start early!!! \pause
\item Go through last year projects \pause
\item Choose your projects \pause
\item Contribute Code \pause
\item Write an excellent proposal and Submit it \pause
\item Interact with the community
\item Be in touch with your mentor, keep on contributing code \pause
\item The D-Day \pause
\item Flip bits, Not Burgers 
\end{itemize}
\end{frame}

\begin{frame}{Doubts and Myths}
\begin{itemize}
\pause
\item Can I get in? \pause
\item But I am not a super cool coder 
\end{itemize}
\end{frame}

\begin{frame}{What Organizations want from Students?}
\begin{itemize}
\pause
\item Passion \pause
\item Basic Programming skills though it may vary from project to project \pause
\item Soft Skills \pause
\item Long Term Commitment
\end{itemize}
\end{frame}

\begin{frame}{After GSoc?}
\pause
You are Awesomer!!!!
\end{frame}

\section{Becoming a mentor?}
\begin{frame}{Open Source communities cannot thrive without...}
\pause
mentors
\end{frame}

\begin{frame}{Who is a mentor?}
\pause
Anyone who is passing down his/her experience to new and existing community members
\end{frame}


\begin{frame}{You can become a mentor to}
\pause
\begin{itemize}
\item include new members into the project \pause
\item get some help to finish off a project for which you dont have the time \pause
\item share your knowledge \pause
\end{itemize}
\end{frame} 

\section{The Good, The Bad and The Ugly}
\begin{frame}{The Good}
\pause
\begin{itemize}
\item A 100\% success in the 2014 edition \pause
\item A lot of code pushed upstream (TSP, OSEM) \pause
\item A lot of enthusiasm shown by mentors and students alike \pause
\item Collaboration with other projects (ownCloud, Zorp etc) \pause
\end{itemize}
\end{frame}


\section{The Bad}
\begin{frame}{The Bad}
\pause
\begin{itemize}
\item Mentor's not getting enough recognition \pause
\item The fine line over flexibility and too much red tape \pause
\end{itemize}
\end{frame}

\section{The Ugly}
\begin{frame}{The Ugly}
\pause
\begin{itemize}
\item The 'Vouch' Fiasco \pause
\end{itemize}
\end{frame}

\section{The Future}
\begin{frame}{The Future}
\pause
\begin{itemize}
\item Google Code-In 2015
\item Google Summer of Code 2015
\end{itemize}
\end{frame}

\begin{frame}{Google Code-In}
\pause
\begin{itemize}
\item Targets school students
\item Different Goals
\pause
\begin{itemize}
\item Code
\item Documentation/Training
\item Outreach
\item QA
\item User Interface
\end{itemize}
\end{itemize}
\end{frame}

\section{Questions???}
\pause
\begin{frame}{Questions???}
\end{frame}

\end{document} 
